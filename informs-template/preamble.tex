%% ==========================================================================
%% preamble.tex - Additional packages for INFORMS papers
%% ==========================================================================
%% Include with: %% ==========================================================================
%% preamble.tex - Core LaTeX packages for academic papers
%% ==========================================================================
%% Include with: %% ==========================================================================
%% preamble.tex - Core LaTeX packages for academic papers
%% ==========================================================================
%% Include with: %% ==========================================================================
%% preamble.tex - Core LaTeX packages for academic papers
%% ==========================================================================
%% Include with: \input{preamble.tex}
%% For TikZ figures, also include: \input{tikz/setup.tex}
%% ==========================================================================

%%% Core math and graphics
\usepackage{amsmath,amssymb,amsfonts}
\usepackage{amsthm}             % theorem environments (proof, etc.)
\usepackage{mathtools}          % extends amsmath
\usepackage{mathrsfs}           % \mathscr for script fonts
\usepackage{bm}                 % bold math symbols
\usepackage{graphicx}
\usepackage{xcolor}             % color support

%%% Typography
\usepackage{microtype}          % better typography (kerning, spacing)
\usepackage[english]{babel}     % hyphenation patterns
\usepackage{setspace}           % line spacing (\onehalfspacing, \doublespacing)

%%% Tables
\usepackage{booktabs}           % professional tables (\toprule, \midrule, \bottomrule)
\usepackage{multirow}           % multi-row cells in tables
\usepackage{array}              % extended column definitions

%%% Figures
\usepackage{caption}
\captionsetup{
  justification=raggedright,
  singlelinecheck=false
}
\usepackage{subcaption}         % subfigures (\begin{subfigure})

%%% Algorithms (comment out if not needed)
\usepackage[ruled,vlined]{algorithm2e}

%%% Document structure
\usepackage{subfiles}           % multi-file documents
\usepackage[toc,page,title,titletoc]{appendix}

%%% Hyperlinks (load near end - redefines many commands)
\PassOptionsToPackage{hyphens}{url}  % allow URL line breaks at hyphens
% Define custom link colors (dark blue scheme, print-friendly)
\definecolor{linkblue}{rgb}{0.0, 0.0, 0.55}      % dark navy for internal refs
\definecolor{citeblue}{rgb}{0.0, 0.0, 0.55}      % dark navy for citations
\definecolor{urlblue}{rgb}{0.0, 0.25, 0.55}      % subtle teal for URLs
\usepackage{hyperref}
\hypersetup{
  colorlinks=true,
  linkcolor=linkblue,
  citecolor=citeblue,
  urlcolor=urlblue,
  plainpages=false,
  hypertexnames=false
}
\usepackage{bookmark}           % better PDF bookmarks (must follow hyperref)

%% For TikZ figures, also include: %% ==========================================================================
%% tikz/setup.tex - TikZ and pgfplots packages
%% ==========================================================================
%% Include with: \input{tikz/setup.tex}
%% Remove this input if your paper doesn't use TikZ figures.
%% ==========================================================================

%%% TikZ and plotting
\usepackage{tikz}
\usetikzlibrary{shapes,shapes.arrows,arrows,arrows.meta,positioning,calc,patterns,decorations.pathreplacing,topaths,automata}
\usepackage{pgfplots}
\usepgfplotslibrary{groupplots,dateplot}
\usepackage{pgfplotstable}
\pgfplotsset{compat=1.18}

%%% Add more TikZ libraries as needed:
% \usetikzlibrary{backgrounds}
% \usetikzlibrary{fit}
% \usetikzlibrary{matrix}
% \usetikzlibrary{chains}

%% ==========================================================================

%%% Core math and graphics
\usepackage{amsmath,amssymb,amsfonts}
\usepackage{amsthm}             % theorem environments (proof, etc.)
\usepackage{mathtools}          % extends amsmath
\usepackage{mathrsfs}           % \mathscr for script fonts
\usepackage{bm}                 % bold math symbols
\usepackage{graphicx}
\usepackage{xcolor}             % color support

%%% Typography
\usepackage{microtype}          % better typography (kerning, spacing)
\usepackage[english]{babel}     % hyphenation patterns
\usepackage{setspace}           % line spacing (\onehalfspacing, \doublespacing)

%%% Tables
\usepackage{booktabs}           % professional tables (\toprule, \midrule, \bottomrule)
\usepackage{multirow}           % multi-row cells in tables
\usepackage{array}              % extended column definitions

%%% Figures
\usepackage{caption}
\captionsetup{
  justification=raggedright,
  singlelinecheck=false
}
\usepackage{subcaption}         % subfigures (\begin{subfigure})

%%% Algorithms (comment out if not needed)
\usepackage[ruled,vlined]{algorithm2e}

%%% Document structure
\usepackage{subfiles}           % multi-file documents
\usepackage[toc,page,title,titletoc]{appendix}

%%% Hyperlinks (load near end - redefines many commands)
\PassOptionsToPackage{hyphens}{url}  % allow URL line breaks at hyphens
% Define custom link colors (dark blue scheme, print-friendly)
\definecolor{linkblue}{rgb}{0.0, 0.0, 0.55}      % dark navy for internal refs
\definecolor{citeblue}{rgb}{0.0, 0.0, 0.55}      % dark navy for citations
\definecolor{urlblue}{rgb}{0.0, 0.25, 0.55}      % subtle teal for URLs
\usepackage{hyperref}
\hypersetup{
  colorlinks=true,
  linkcolor=linkblue,
  citecolor=citeblue,
  urlcolor=urlblue,
  plainpages=false,
  hypertexnames=false
}
\usepackage{bookmark}           % better PDF bookmarks (must follow hyperref)

%% For TikZ figures, also include: %% ==========================================================================
%% tikz/setup.tex - TikZ and pgfplots packages
%% ==========================================================================
%% Include with: %% ==========================================================================
%% tikz/setup.tex - TikZ and pgfplots packages
%% ==========================================================================
%% Include with: \input{tikz/setup.tex}
%% Remove this input if your paper doesn't use TikZ figures.
%% ==========================================================================

%%% TikZ and plotting
\usepackage{tikz}
\usetikzlibrary{shapes,shapes.arrows,arrows,arrows.meta,positioning,calc,patterns,decorations.pathreplacing,topaths,automata}
\usepackage{pgfplots}
\usepgfplotslibrary{groupplots,dateplot}
\usepackage{pgfplotstable}
\pgfplotsset{compat=1.18}

%%% Add more TikZ libraries as needed:
% \usetikzlibrary{backgrounds}
% \usetikzlibrary{fit}
% \usetikzlibrary{matrix}
% \usetikzlibrary{chains}

%% Remove this input if your paper doesn't use TikZ figures.
%% ==========================================================================

%%% TikZ and plotting
\usepackage{tikz}
\usetikzlibrary{shapes,shapes.arrows,arrows,arrows.meta,positioning,calc,patterns,decorations.pathreplacing,topaths,automata}
\usepackage{pgfplots}
\usepgfplotslibrary{groupplots,dateplot}
\usepackage{pgfplotstable}
\pgfplotsset{compat=1.18}

%%% Add more TikZ libraries as needed:
% \usetikzlibrary{backgrounds}
% \usetikzlibrary{fit}
% \usetikzlibrary{matrix}
% \usetikzlibrary{chains}

%% ==========================================================================

%%% Core math and graphics
\usepackage{amsmath,amssymb,amsfonts}
\usepackage{amsthm}             % theorem environments (proof, etc.)
\usepackage{mathtools}          % extends amsmath
\usepackage{mathrsfs}           % \mathscr for script fonts
\usepackage{bm}                 % bold math symbols
\usepackage{graphicx}
\usepackage{xcolor}             % color support

%%% Typography
\usepackage{microtype}          % better typography (kerning, spacing)
\usepackage[english]{babel}     % hyphenation patterns
\usepackage{setspace}           % line spacing (\onehalfspacing, \doublespacing)

%%% Tables
\usepackage{booktabs}           % professional tables (\toprule, \midrule, \bottomrule)
\usepackage{multirow}           % multi-row cells in tables
\usepackage{array}              % extended column definitions

%%% Figures
\usepackage{caption}
\captionsetup{
  justification=raggedright,
  singlelinecheck=false
}
\usepackage{subcaption}         % subfigures (\begin{subfigure})

%%% Algorithms (comment out if not needed)
\usepackage[ruled,vlined]{algorithm2e}

%%% Document structure
\usepackage{subfiles}           % multi-file documents
\usepackage[toc,page,title,titletoc]{appendix}

%%% Hyperlinks (load near end - redefines many commands)
\PassOptionsToPackage{hyphens}{url}  % allow URL line breaks at hyphens
% Define custom link colors (dark blue scheme, print-friendly)
\definecolor{linkblue}{rgb}{0.0, 0.0, 0.55}      % dark navy for internal refs
\definecolor{citeblue}{rgb}{0.0, 0.0, 0.55}      % dark navy for citations
\definecolor{urlblue}{rgb}{0.0, 0.25, 0.55}      % subtle teal for URLs
\usepackage{hyperref}
\hypersetup{
  colorlinks=true,
  linkcolor=linkblue,
  citecolor=citeblue,
  urlcolor=urlblue,
  plainpages=false,
  hypertexnames=false
}
\usepackage{bookmark}           % better PDF bookmarks (must follow hyperref)

%% For TikZ figures, also include: %% ==========================================================================
%% tikz/setup.tex - TikZ and pgfplots packages
%% ==========================================================================
%% Include with: %% ==========================================================================
%% tikz/setup.tex - TikZ and pgfplots packages
%% ==========================================================================
%% Include with: %% ==========================================================================
%% tikz/setup.tex - TikZ and pgfplots packages
%% ==========================================================================
%% Include with: \input{tikz/setup.tex}
%% Remove this input if your paper doesn't use TikZ figures.
%% ==========================================================================

%%% TikZ and plotting
\usepackage{tikz}
\usetikzlibrary{shapes,shapes.arrows,arrows,arrows.meta,positioning,calc,patterns,decorations.pathreplacing,topaths,automata}
\usepackage{pgfplots}
\usepgfplotslibrary{groupplots,dateplot}
\usepackage{pgfplotstable}
\pgfplotsset{compat=1.18}

%%% Add more TikZ libraries as needed:
% \usetikzlibrary{backgrounds}
% \usetikzlibrary{fit}
% \usetikzlibrary{matrix}
% \usetikzlibrary{chains}

%% Remove this input if your paper doesn't use TikZ figures.
%% ==========================================================================

%%% TikZ and plotting
\usepackage{tikz}
\usetikzlibrary{shapes,shapes.arrows,arrows,arrows.meta,positioning,calc,patterns,decorations.pathreplacing,topaths,automata}
\usepackage{pgfplots}
\usepgfplotslibrary{groupplots,dateplot}
\usepackage{pgfplotstable}
\pgfplotsset{compat=1.18}

%%% Add more TikZ libraries as needed:
% \usetikzlibrary{backgrounds}
% \usetikzlibrary{fit}
% \usetikzlibrary{matrix}
% \usetikzlibrary{chains}

%% Remove this input if your paper doesn't use TikZ figures.
%% ==========================================================================

%%% TikZ and plotting
\usepackage{tikz}
\usetikzlibrary{shapes,shapes.arrows,arrows,arrows.meta,positioning,calc,patterns,decorations.pathreplacing,topaths,automata}
\usepackage{pgfplots}
\usepgfplotslibrary{groupplots,dateplot}
\usepackage{pgfplotstable}
\pgfplotsset{compat=1.18}

%%% Add more TikZ libraries as needed:
% \usetikzlibrary{backgrounds}
% \usetikzlibrary{fit}
% \usetikzlibrary{matrix}
% \usetikzlibrary{chains}

%%
%% The informs4.cls already includes:
%%   - Math: amsmath, amssymb, amsfonts, bm
%%   - Fonts: tgtermes, newtxtext, newtxmath
%%   - Bibliography: natbib
%%   - Hyperlinks: hyperref, bookmark
%%   - Other: graphicx, xcolor, array, url
%%   - Theorem environments: defined by the class
%%   - Operators: \argmin, \argmax
%% ==========================================================================

%%% Additional math packages (not in cls)
\usepackage{mathtools}          % extends amsmath (dcases, coloneqq, etc.)
% Note: mathrsfs conflicts with newtx fonts; use \mathcal for script fonts

%%% Tables
\usepackage{booktabs}           % professional tables (\toprule, \midrule, \bottomrule)
\usepackage{multirow}           % multi-row cells in tables

%%% Figures
\usepackage{subcaption}         % subfigures (\begin{subfigure})

%%% Algorithms (comment out if not needed)
% \usepackage[ruled,vlined]{algorithm2e}

%% ==========================================================================
%% Common math macros
%% ==========================================================================

%%% Number sets
\newcommand{\R}{\mathbb{R}}     % real numbers
\newcommand{\Z}{\mathbb{Z}}     % integers
\newcommand{\N}{\mathbb{N}}     % natural numbers
\newcommand{\Q}{\mathbb{Q}}     % rationals
\newcommand{\C}{\mathbb{C}}     % complex numbers

%%% Probability and expectation
\newcommand{\E}{\mathbb{E}}     % expectation
\newcommand{\Var}{\mathrm{Var}} % variance
\newcommand{\Cov}{\mathrm{Cov}} % covariance
\newcommand{\pr}{\mathbb{P}}    % probability
\newcommand{\ind}[1]{\mathbf{1}_{\{#1\}}} % indicator function

%%% Common operators (argmin, argmax already defined in informs4.cls)

%%% Brackets and norms
\newcommand{\abs}[1]{\left|#1\right|}           % absolute value
\newcommand{\norm}[1]{\left\|#1\right\|}        % norm
\newcommand{\inner}[2]{\langle#1,#2\rangle}     % inner product
\newcommand{\floor}[1]{\lfloor#1\rfloor}        % floor
\newcommand{\ceil}[1]{\lceil#1\rceil}           % ceiling
\newcommand{\pos}[1]{\left[#1\right]^+}         % positive part

%%% Convergence
\newcommand{\dto}{\Rightarrow}                  % convergence in distribution
\newcommand{\pto}{\xrightarrow{p}}              % convergence in probability
\newcommand{\asto}{\xrightarrow{a.s.}}          % almost sure convergence
